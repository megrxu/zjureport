\documentclass{zjureport}

\usepackage{hyperref}
\hypersetup{
    colorlinks=true,
}

\major{信息工程}
\name{张扬}
\title{本科实验报告}
\stuid{3171002333}
\college{信息与电子工程学院}
\date{\zhtoday}
\lab{东四-A-308}
\course{\LaTeX 模板使用}
\instructor{赵林平}
\grades{93}
\expname{使用实验报告模板}
\exptype{操作实验}
\partner{无}

\begin{document}

\makecover % 如果不想要封面,可以注释掉这一行
\makeheader

\section{实验目的和要求}
本文档使用了笔者编写的适用于 \LaTeX 的浙江大学实验报告模板,用于展示最终达到的效果。出于个人喜好,目前推荐的引擎是 \href{https://github.com/tectonic-typesetting/tectonic/}{Textonic}。

由于浙江大学近来更改了其\href{https://www.zju.edu.cn/572/list.htm}{整体视效设计},包括``浙江大学''四字,本模板采用了最新的 Logo ,因此可能和课程中老师提供的 Microsoft Word 模板中的 Logo 有细微区别。

\section{实验内容和步骤}

\subsection{实验内容}

\subsection{实验步骤}

\section{主要仪器设备}
计算机及配套软件。

\section{操作方法和实验步骤}

\section{实验数据记录和处理}

\begin{lstlisting}[language=C]
#include <stdio.h>
int main() {   
    int number;
   
    printf("Enter an integer: ");  
    
    // reads and stores input
    scanf("%d", &number);

    // displays output
    printf("You entered: %d", number);
    
    return 0;
}
\end{lstlisting}

\section{实验结果与分析}

\end{document}
